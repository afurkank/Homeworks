\documentclass[10pt,a4paper, margin=1in]{article}
\usepackage{fullpage}
\usepackage{amsfonts, amsmath, pifont}
\usepackage{amsthm}
\usepackage{graphicx}
\usepackage{float}
\usepackage{mathtools}
\usepackage{listings}
\usepackage{tkz-euclide}
\usepackage{tikz}
\usepackage{pgfplots}
\pgfplotsset{compat=1.13}
\usepackage{pdfpages}
\usepackage{geometry}
 \geometry{
 a4paper,
 total={210mm,297mm},
 left=10mm,
 right=10mm,
 top=10mm,
 bottom=10mm,
 }
 % Write both of your names here. Fill exxxxxxx with your ceng mail address.
 \author{
  Karacık, Ahmet Furkan\\
  \texttt{e2310209@ceng.metu.edu.tr}
  \and
  LastName2, FirstName2\\
  \texttt{exxxxxxx@ceng.metu.edu.tr}
}

\title{CENG 384 - Signals and Systems for Computer Engineers \\
Spring 2023 \\
Homework 4}
\begin{document}
\maketitle



\noindent\rule{19cm}{1.2pt}

\begin{enumerate}

\item %write the solution of q1
	\begin{enumerate}
    % Write your solutions in the following items.
    \item %write the solution of q1a

\[H(jw) = \frac{jw-1}{jw+1}\]

\[H(jw) = \frac{Y(jw)}{X(jw)} = \frac{jw-1}{jw+1}\]
\[\Rightarrow jwY(jw) + Y(jw) = jwX(jw) - X(jw)\]
\[\Rightarrow \frac{dy(t)}{dt} + y(t) = \frac{dx(t)}{dt} - x(t)\]

\item
\[h(t) \xleftrightarrow{FT} H(jw)\]
\[\frac{jw-1}{jw+1} = \frac{jw}{1 + jw} - \frac{1}{1 + jw} \xleftrightarrow{FT} \frac{d(e^{-t}u(t)}{dt} - e^{-t}u(t)\]
\[= 2e^{-t}u(t)\] where \(t > 0\)
    \item %write the solution of q1c
\[y = y_h + y_p\]
\[y_h = Ae^{st} \Rightarrow Ase^{st} + Ae^{st} = 0 \Rightarrow Ae^{st}(s+1) = 0 \Rightarrow s = -1\]
\[\Rightarrow y_{h}(t) = Ae^{-t}\]
\[y_p = Kx(t) = Ke^{-2t}u(t) \Rightarrow -2Ke^{-2t} + Ke^{-2t} = -2e^{-2t} + e^{-2t}\]
\[\Rightarrow -Ke^{-2t} = -e^{-2t} \Rightarrow K = 1\]
\[\Rightarrow y_{p}(t) = x(t)\]
\[y(t) = Ae^{-t} + e^{-2t}\]
\[y(0) = 0 \Rightarrow A + 1 = 0 \Rightarrow A = -1\]
\[\Rightarrow y(t) = (e^{-2t} - e^{-t})u(t)\]
	\item\\ %write the solution of q1d
$ $

\begin{center}
	\tikzset{%
		block/.style    = {draw, thick, rectangle, minimum height = 3em,
			minimum width = 3em},
		sum/.style      = {draw, circle, node distance = 2cm}, % Adder
		input/.style    = {coordinate}, % Input
		output/.style   = {coordinate} % Output
	}
	% Defining string as labels of certain blocks.
	\newcommand{\suma}{\Large$+$}
	\newcommand{\inte}{$\displaystyle \int$}
	\newcommand{\derv}{\huge$\frac{d}{dt}$}

	\begin{tikzpicture}[auto, thick, node distance=2cm, >=triangle 45]
	\draw
	node at (0, 0) [input] (inp) {\Large \textopenbullet}
	node [output, right of=inp] (temp1) {}
	node [output, right of=temp1] (t1) {}
	node [output, right of=t1] (t2) {}
	node [output, below of=temp1] (temp2) {}
	node [sum, right of=temp2] (sum1) {\suma}
	node [block, right of=sum1] (int) {\inte}
	node [output, below of=sum1] (temp3) {}
	node [output, right of=temp3] (s1) {}
	node [output, right of=s1] (s2) {}
	node [output, right of=s2] (temp4) {}
	node [sum, right of=t2] (sum2) {\suma}
	node [output, below of=sum2] (temp5) {}
	node [output, right of=sum2] (out) {}
	node [output, right of=out] (out2) {\Large \textopenbullet}
	;
	\draw[-](inp) -- node{$x(t)$} (temp1);
	\draw[->](temp1) -- node{} (sum2);
	\draw[-](sum2) -- node{} (out);
	\draw[-](out) -- node{} (temp4);
	\draw[-](temp4) -- node{$-1$} (temp3);
	\draw[-](temp4) -- node{} (out);
	\draw[->](temp3) -- node{} (sum1);
	\draw[-](temp1) -- node{$-1$} (temp2);
	\draw[->](temp2) -- node{} (sum1);
	\draw[->](sum1) -- node{} (int);
	\draw[-](int) -- node{} (temp5);
	\draw[->](temp5) -- node{} (sum2);
	\draw[->](out) -- node{$y(t)$} (out2);
	\end{tikzpicture}
\end{center}
    \end{enumerate}

\item %write the solution of q2
	\begin{enumerate}
    % Write your solutions in the following items.
    \item %write the solution of q2a
\[y[n] = \xleftrightarrow{FT} Y(e^{jw}) \Rightarrow y[n+1] = e^{jw}Y(e^{jw}) \]
\[ x[n] = \xleftrightarrow{FT} X(e^{jw}) \Rightarrow x[n+1] = e^{jw}X(e^{jw}) \]
\[ \Rightarrow e^{jw}Y(e^{jw}) - \frac{1}{2}Y(e^{jw}) = e^{jw}X(e^{jw}) \]
\[ \Rightarrow Y(e^{jw})(e^{jw} - \frac{1}{2}) = e^{jw}X(e^{jw}) \]
\[ H(e^{jw}) = \frac{Y(e^{jw})}{X(e^{jw})} = \frac{2e^{jw}}{2e^{jw}-1} = \frac{1}{1-\frac{1}{2}e^{-jw}} \]
Divided all terms with \( 2e^{jw} \).

    \item %write the solution of q2b
From the table 5.2, we see that the inverse transform of  \( H(e^{jw}) \) gives us:
\[ h[n] = (\frac{1}{2})^{n}u[n] \]
	\item %write the solution of q2c
\[ y[n] = x[n] \ast h[n] = \sum_{k=-\infty}^{\infty}x[k]h[n-k] = \sum_{k=-\infty}^{\infty}(\frac{3}{4})^{k}u[k](\frac{1}{2})^{n-k}u[n-k] \]
\[ = \sum_{k=0}^{n}(\frac{3}{2})^{k}(\frac{1}{2})^{k}(\frac{1}{2})^{n}2^{k} \]
\[ \sum_{k=0}^{n}(\frac{3}{2})^{k}(\frac{1}{2})^{n} = (\frac{1}{2})^{n}\sum_{k=0}^{n}(\frac{3}{4})^{k} \]
\[ = (\frac{1}{2})^{n}\frac{1-(\frac{3}{4})^{n+1}}{1 - \frac{3}{4}} \]
\[ = (\frac{1}{2})^{n}(4 - \frac{3^{n+1}}{4^n}) \]

    \end{enumerate}

\item %write the solution of q3
	\begin{enumerate}
    % Write your solutions in the following items.
    \item %write the solution of q3a
Let \(G(jw) = X(jw)H_{1}(jw)  \) and \( Y(jw) = G(jw)H_{2}(jw) \)
Then,
\[Y(jw) = X(jw)H_{1}(jw)H_{2}(jw) = X(jw)\frac{1}{jw+1}\frac{1}{jw+2} \]
\[ \Rightarrow Y(jw)(jw+1)(jw+2) = X(jw) \]
\[ \Rightarrow (jw)^{2}Y(jw) + 3jwY(jw) + 2Y(jw) = X(jw) \]
\[\Rightarrow y^{\prime\prime}(t) + 3y^{\prime}(t) + 2y(t) = x(t)  \]

    \item %write the solution of q3b
From \( Y(jw) = X(jw)H_{1}(jw)H_{2}(jw) \)  we see that if   \( Y(jw) = H(jw)X(jw) \)
Then,
\[ H(jw) = H_{1}(jw)H_{2}(jw) \]
Let \(x(t) \xleftrightarrow{FT} H_{1}(jw) = \frac{1}{jw+1} \)  and   \( y(t) \xleftrightarrow{FT} H_{2}(jw) = \frac{1}{jw+2} \)
Then,
\[ x(t) = e^{-t}u(t) \]
\[ y(t) = e^{-2t}u(t) \]
We also see that
\[ (x \ast y)(t) \xleftrightarrow{FT} H_{1}(jw)H_{2}(jw) = H(jw) \]
It is clear that if we find \(H(jw)\) we can find \(h(t)\) using Inverse Fourier Transform. To obtain \(H(jw)\) we can use the convolution of input and output signals.

\[ (x \ast y)(t) = \int_{-\infty}^{\infty}x(\tau)y(t-\tau)d\tau \]
\[ = \int_{-\infty}^{\infty}e^{-\tau}u(\tau)e^{-2(t - \tau)}u(t-\tau)d\tau \]
\[ = e^{-2t}\int_{0}^{t}e^{-\tau}e^{2\tau}d\tau  = \left. e^{-2t}(e^{\tau}) \right|^{t}_{0}   \]
\[\Rightarrow h(t) = (e^{-t} - e^{-2t})u(t) \]

	\item %write the solution of q3c
\[ Y(jw) = X(jw)H(jw) = jwH(jw) \]
\[y(t) \xleftrightarrow{FT} Y(jw) \]
\[ \frac{dh(t)}{dt} \xleftrightarrow{FT} jwH(jw) \]
\[\Rightarrow y(t) = \frac{dh(t)}{dt} = (-e^{-t} + 2e^{-2t})u(t) \]

    \end{enumerate}

\item %write the solution of q4
    \begin{enumerate}
    % Write your solutions in the following items.
    \item %write the solution of q4a
\[Y(e^{jw}) = X(e^{jw})(H_{1}(e^{jw}) + H_{2}(e^{jw}))   \]
\[ = X(e^{jw})(\frac{3}{3 + e^{-jw}} + \frac{2}{2 + e^{-jw}})  \]
\[ = X(e^{jw})(\frac{ 12 + 5e^{-jw} }{ 6 + 5e^{-jw} + e^{-2jw} })  \]
\[ \Rightarrow e^{-2jw}Y(e^{jw}) + 5e^{-jw}Y(e^{jw}) + 6Y(e^{jw}) = 5e^{-jw}X(e^{jw}) + 12X(e^{jw})  \]
\[ \Rightarrow y[n-2] + 5y[n-1] + 6y[n] = 5x[n-1] + 12x[n]  \]
    \item %write the solution of q4b
\[ H(e^{jw}) = \frac{ Y(e^{jw}) }{ X(e^{jw}) } = \frac{ 12 + 5e^{-jw} }{ 6 + 5e^{-jw} + e^{-2jw} }  \]

	\item %write the solution of q4c
\[ h[n] \xleftrightarrow{FT} H(e^{jw}) = \frac{3}{3+e^{-jw}} + \frac{2}{ 2+e^{-jw} } = \frac{1}{ 1 + \frac{1}{3}e^{-jw} } + \frac{1}{ 1 + \frac{1}{2}e^{-jw} }  \]

    \end{enumerate}
\[ \Rightarrow h[n] = (-\frac{1}{3})^{n}u[n] + (-\frac{1}{2})^{n}u[n]  \]
\item %write the solution of q5

    Frequency domain magnitude plot of encoded message:

    \includegraphics[scale=0.7]{freq_encoded.png}\\

    Frequency domain magnitude plot of decoded message:

    \includegraphics[scale=0.7]{freq_decoded.png}\\

    Time domain magnitude plot of encoded message:

    \includegraphics[scale=0.7]{time_encoded.png}\\

    Time domain magnitude plot of decoded message:

    \includegraphics[scale=0.7]{time_decoded.png}\\

    Code:\\

    \begin{lstlisting}[language=Python]

    import numpy as np
    import matplotlib.pyplot as plt
    from scipy.io import wavfile
    import scipy.io

    # fast fourier transformation function
    def fft(x):
        x = np.asarray(x, dtype=float)
        N = x.shape[0]
        n = np.arange(N)
        k = n.reshape((N,1))
        M = np.exp(-2j * np.pi * k * n / N)
        return np.dot(M, x)

    # inverse fast fourier transformation function
    def ifft(X):
        X = np.asarray(X, dtype=complex)
        N = X.shape[0]
        n = np.arange(N)
        k = n.reshape((N,1))
        M = np.exp(2j * np.pi * k * n / N)
        return np.dot(M, X) / N

    # get data
    path = r"/content/encoded.wav"
    samplerate, data = wavfile.read(path)
    length = data.shape[0] / samplerate

    # obtain fourier domain representation of the signal
    fourier_domain_signal = fft(data)

    # apply decoding recipe step 2
    X_prime_1 = fourier_domain_signal[:len(fourier_domain_signal)//2]
    X_prime_2 = fourier_domain_signal[len(fourier_domain_signal)//2:]

    X_1 = []
    for i in range(1, len(X_prime_1)+1):
        X_1.append(X_prime_1[-i])
    X_2 = []
    for i in range(1, len(X_prime_2)+1):
        X_2.append(X_prime_2[-i])

    X_prime = np.concatenate((X_1, X_2), axis=None)

    # return to time domain
    time_domain_signal = ifft(X_prime)

    # convert signal to wav file
    wavfile.write("decoded_message.wav", samplerate, time_domain_signal.astype(np.int16))

    # plot function
    def plot(data_to_plot):
        time = np.linspace(0., length, data_to_plot.shape[0])

        plt.plot(time, data_to_plot[:], label="signal")

        plt.legend()

        plt.xlabel("Time [s]")

        plt.ylabel("Amplitude")

        plt.show()


    \end{lstlisting}\\




\end{enumerate}


\end{document}
